%TEX root = ../dissertation.tex

\iffalse \bibliography{bibliography/dissertation} \fi

\chapter{Designing a System for Understanding Programs}
\label{chapter:pegd}

\section{Design Principles} 

Programming is hard to learn. Unfortunately, as shown in the previous section, few programming languages or environments are designed to introduce the basics of good software engineering practice. Specially in Architecture, the environments used to support \gls{gd} either they are old or obsolete, or they enforce particular programming methods that are inadequate, or they are not pedagogic, meaning that they are not designed for the particular programming skills of the \gls{gd} community.

The literature is vast, but it is clearly divided in two perspectives: software engineering perspective, or psychological/educational perspective. That means either systems designed for expert programmers~\cite{carlson2005eclipse,intellij2001intellij,lighttable,boudreau2002netbeans,guckenheimer2006software}, or systems designed for novices~\cite{papert1980mindstorms,goldberg1983smalltalk,GuoSIGCSE2013,reas2006processing}. The problem is that while programing systems designed for expert programmers are constantly being updated with the latest features, the systems designed for novices, despite of proposing interesting features, they are usually obsolete or in disuse. It negatively affects other communities that are not focused on develop software for industry, mainly their users which are in general novices.

For example the \gls{gd} community is focused on build software to support Architecture methods, however designers are in desperate need of a programming system that correspond to their needs. These needs are from different natures for instance: adoption of novices, deal with increasingly complexity of \gls{gd} programs, tools that allow to test ideas quickly.

In the next section we propose 