%!TEX root = ../dissertation.tex

\begin{otherlanguage}{portuguese}
% \begin{abstract}
% \thispagestyle{plain}
% O teu resumo aqui...

% % Keywords
% \begin{flushleft}

% \palavrasChave{as tuas palavras chave}

% \end{flushleft}

% \end{abstract}

\chapter*{Resumo}
\thispagestyle{empty}

Nos dias de hoje verifica-se uma clara evolução nas ferramentas utilizadas pelos arquitectos, havendo uma substituição dos processos arquitectónicos tradicionais e das codificações arquitecturais clássicas para uma nova área denominada Desenho Generativo. Desenho Generativo (DG) consiste na aplicação de métodos computacionais na geração de objectos arquitecturais. Nesta área, designers escrevem programas que quando executados produzem modelos geométricos. Este movimento é claramente visível tanto a nível académico, com o actual currículo de arquitectura que adopta cursos de programação, como na indústria com ateliers de design a substituírem os processos tradicionais por aplicações informáticas. No entanto, para a maioria dos utilizadores, a transição dos processos tradicionais para os métodos de desenho generativo pode ser significativamente difícil, pelo que é importante que os sistema de DG encorajem os utilizadores a aprenderem, dando-lhes o suporte necessário para evoluirem o seus conhecimentos e estratégias. Infelizmente, a maioria dos sistemas não é capaz de responder a esta necessidade por serem tecnologias antigas e obsoletas ou que não imponhem qualquer bom princípio da engenharia de software, tal como a documentação do código. Para ultrapassar este problema, esta tese propõe duas ferramentas de programação, nomeadamente (1) uma ferramenta de correlação-programa que permite aos arquitectos combinarem código com esquissos arquitecturais e (2) uma ferramenta de feedback imediato que acelera o efeito das acções feitas no cógido imediatamente mostrando-as no output do programa. Verificámos que a primeira ferramenta torna a documentação do programa numa tarefa menos cansativa, minimizando a falta de documentação dos programas de DG e que a segunda ferramenta melhora a visualização do programa criando um novo meio de auxílio que ajuda o desenvolvimento de programas. Esta tese utiliza a ferramenta Rosetta para implementar e validar as ferramentas propostas, mostrando claramente as vantagens deste ambiente de programação moderno sobre a maioria dos sistemas utilizados.

\begin{flushleft}
\palavrasChave{feedback imediato; correlação com esquissos; DrRacket; Rosetta}
\end{flushleft}

\end{otherlanguage}
