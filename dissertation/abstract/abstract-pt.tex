%!TEX root = ../dissertation.tex

\begin{otherlanguage}{portuguese}
% \begin{abstract}
% \thispagestyle{plain}
% O teu resumo aqui...

% % Keywords
% \begin{flushleft}

% \palavrasChave{as tuas palavras chave}

% \end{flushleft}

% \end{abstract}

\chapter*{Resumo}
\thispagestyle{empty}

Nos dias de hoje verifica-se uma evolução nas ferramentas utilizadas pelos arquitectos, havendo uma substituição dos processos arquitectónicos tradicionais e das formas clássicas de código arquitectural por uma área moderna denominada de Desing Generativo. Esta consiste na aplicação de métodos computacionais para projectar estruturas arquitectónicas ou objectos. Nesta área os designers escrevem programas que quando executados produzem modelos geométricos. Este movimento é claramente visível tanto a nível académico, com o actual currículo de arquitectura que adopta cursos de programação, como na indústria com ateliers de design a substituírem os processos tradicionais por aplicações informáticas. No entanto, para a maioria dos utilizadores, a transição das aplicações para métodos pode ser significativamente difícil, pelo que é importante que os sistema encorajem os utilizadores a aprender, dando-lhes o suporte necessário para uma evolução do conhecimento e da estratégia. Infelizmente, a maioria dos sistemas não é capaz de responder a essa necessidade devido a serem tecnologias antigas e obsoletas ou que não imponham qualquer  principio de engenharia de software, tal como a documentação do código. Para ultrapassar este problema, esta tese propõe duas ferramentas de programação, nomeadamente (1) uma ferramenta de correlação-programa que permite aos arquitectos combinar o código com o desenho e (2) uma ferramenta de feedback imediato que acelera o efeito das acções de output do programa. Verificámos que a primeira ferramenta torna a documentação do programa numa tarefa menos cansativa minimizando a falta de documentação no programa e que a segunda ferramenta melhora a visualização do programa criando um novo meio de auxílio que ajuda a conceber programas. Esta tese utiliza a ferramenta Rosetta para implementar e validar as ferramentas propostas, mostrando claramente as vantagens deste ambiente de programação moderno sobre a maioria dos sistemas utilizados.

\begin{flushleft}
\palavrasChave{feedback imediato; correlação com images; DrRacket; Rosetta}
\end{flushleft}

\end{otherlanguage}
