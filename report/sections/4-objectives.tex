%!TEX root = ../report.tex

% 
% Objectives
% 

\section{Objectives}

This work addresses two challenges:

\begin{itemize}
\item improve the program comprehension process.
\item minimize the lack of documentation in the programs.
\end{itemize}

To overcome it, we will investigate better ways to help programmers in their conceptual tasks. The objective is to design and implement innovative tools which support and encourage new ways of thinking, and therefore, enabling programmers to see and understand their programs easier.

Our approach to achieve this objective is, at first, analyze how programming tools can be designed to improve the program comprehension. The \textit{Learnable Programming}~\cite{learnableProg,inventingPrin} approach has shown interesting insights in this direction. Secondly, based on this analysis, we will implement two interactive tools: 

\begin{enumerate}
\item \textit{code correlation tool,} which correlates images with code. The source code will be commented with visual images rather than only plain text. This enables programmers to effortlessly understand the code, thereby acquiring a better mental model.
\item \textit{live code execution tool,} which executes the program upon changes. This tool minimizes the latency between the code written and its execution, encouraging users to experiment ideas quickly, and augmenting their comprehension about the program.
\end{enumerate}

This thesis will produce the following expected results:
 
\begin{itemize}
\item[] \textit{Expected results:} \textit{i)} a specification of each tool, its purpose and how this tool is designed to support its purpose; \textit{ii)} an implementation of a prototype; and \textit{iii)} an experimental evaluation with a comparison to other similar tools.
\end{itemize}