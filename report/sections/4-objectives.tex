%!TEX root = ../report.tex

% 
% Objectives
% 

\section{Objectives}

This work addresses two challenges:

\begin{itemize}
\item minimize the lack of documentation in the programs.
\item improve the program comprehension process.
\end{itemize}

To overcome these challenges, we will investigate better ways to help programmers in their conceptual tasks. The goal is to design and implement innovative tools which support and encourage new ways of thinking, and therefore, enabling programmers to more easily see and understand their programs.

Our approach to achieve this objective is to, at first, analyze how programming tools can improve program comprehension. The \textit{Learnable~Programming}~\cite{learnableProg,inventingPrin} approach has shown interesting insights in this direction. Secondly, based on this analysis, and in order to prove our ideas, we will implement two interactive tools tailored for generative design programs: 

\begin{enumerate}
\item \textit{Sketch-program correlation tool,} which will encourage architects and designers to reuse their conceptual sketches, such as that shown in Figure~\ref{fig:sketch}, to visually document their programs. These sketches will be correlated with the program source code in a way that significantly reduces the effort to read the code. Therefore, it allows users to acquire a better mental model.

\item \textit{Immediate feedback tool,} which will give a new medium for architects and designers to create new ideas by continuously reacting with changes in their models. This tool will minimize the latency between writing the code and executing it, consequently this will encourage users to experiment ideas quickly, augmenting their comprehension about the program.
\end{enumerate}

This thesis will produce the following expected results: \textit{i)} a specification of each tool, its purpose and how this tool is designed to support its purpose, \textit{ii)} an implementation of a prototype, and \textit{iii)} an experimental evaluation with a comparison to other similar tools.