%!TEX root = ../report.tex

% 
% Introduction
% 

\section{Introduction (2/3pgs)}

topics (background information)
	-programming becomes used in different areas due to the facilities such as programming languages and programming environments
	-e.g. in architecture area the generative design methods become an important topic of research
	-program documentation is a boring textual description and often is left behind
	-program comprehension means read the code and think as a machine does

	(gap in knowledge)
	-general programming environments either does not address this issue or are targeted for professionals
	-live code environments overcome this problem by providing immediate feedback.
		-Two main problems: (i) this feature alone, worth nothing (ii) targeted for live performers
	-generative design environments sidestep this problem by providing a visual programming language.
		-problem: straightforward for beginners but got complex as the program grows
		-textual environments problems:
			-often provide a poor programming language
			-there is no immediate feedback at all
			-few provide traceability
			-the text editor just supports text

	(Hypothesis/aim)
	-read the code and create by reacting can improve program comprehension and program documentation.
		-enabling beginner programmers to effortless read the code and immediately see the result of their actions.

	(Proposed solution/plan-how it fills the gaps)
	-Based on learning programming (B. Victor et al) and old systems programming (cite) we proposed:
	-two interactive tools, to integrate in a generative design program environment, (1) image in the code correlation and (2) live code
		-(1) read the code effortless
		-(2) experiment ideas instantly 