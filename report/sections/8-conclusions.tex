%!TEX root = ../report.tex

% 
% Conclusions
% 

\section{Conclusions}
\label{sec:fin}

We found that a programming environment can be designed to address the several problems with programming in general. As we proposes, two distinct tools, designed to work cohesively together, eliminate significant barriers in programming. For example, the mechanism which correlates code with image eliminates the difficulty of reading the code. Because images and sketches are more illustrative than just plain text. On the other hand, the mechanism of immediate feedback encourages users to test new ideas and get feedback as fast as possible.

Among the generative design systems which support programming in a textual form~\cite{aish2012designscript,lopes2011portable} just one~\cite{lopes2011portable} supports other elements, in the editor, besides plain text. However, none correlates text with sketches (so common in this area). Moreover, in these systems the only way to get immediate feedback is using a stepwise debugger which stops the entire program execution.

Using the proposed tools the above problem will be resolved. However, as natural, we expect these tools to have some limitations. Firstly, relating with the automatic recognition of symbols in the sketch. Until now, we have had bad experiences with OCR engines. If necessary, and as a proof-of-concept we will generate the OCR data manually. And secondly, similar to all other generative systems, the immediate feedback tool will not scale for arbitrarily complex programs.

A functional prototype of these features was already implemented. In a near future, we will improve the implementation of these mechanisms and, if time permits, we will explore new mechanisms and a possible integration.

More important than these features are the underlying design principles that they represent. Understand how these principles enable the programmer to think is the initial point to evolve the way we programming.