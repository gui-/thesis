%!TEX root = ../report.tex

% 
% Conclusions
% 

\section{Conclusions}

We have shown that we can turn programming even more understandable, flexible and productive. Consequently, these are the main desires of projects whose users are not experts in programming, such as Rosetta. As a result, we proposed an tailored environment for generative design with well designed tools, to be used with Rosetta.

Our solution suggest some tools to integrate into the DrRacket~\ac{ide} which enable the user to read the program helped by traceability and sketches, follow the program flow by using a tracer, and see the state of the program by using immediate feedback. Recalling that, each one of these tools, were designed for help people to understand programs, and also the concept it implements can be used in the most different areas, in our case we will use the Rosetta as a proof-of-concept.

Part of traceability mechanism that relates image with code has already been implemented, users have its sketches associated with its functions. However, some additional work must be done to improve this mechanism, such as automate the arrows binding and also recognizing figure symbols.

In order to implement the others tools, we have to come up with another implementation ideas. The tool's design is just as critical as the tool's implementation. In addition, editing directly the DrRacket~\ac{ide} source code, have been proved as a slow process and will not support the implementation of several ideas, here presented. 

In the future, we will explore new tools as well as new environments to integrate into the Rosetta programming environment.