%!TEX root = ../report.tex

% 
% Evaluation
% 
\section{Evaluation (1/2pgs)}
\label{sec:eval}

The evaluation of the proposed architecture will be performed experimentally, building a prototype. The Rosetta generative design tool will be used as a case of study. Since our tool will contribute for generative design work. Namely, on handling the manuscripts sketches in the code, correlating visual elements with generating designs and giving a platform for experiment ideas.

Furthermore, to evaluate our proposal we suggest the following evaluation metrics.

\begin{itemize}
\item \textbf{Correctness.} To assess the quality of our system we plan to test, individually, each proposed feature with an specific test case scenario, for example using the slider widget to explore the result of a parametric function, inserting different kind of images and check if the image is well correlated with the function's parameters, and so on. 

\item \textbf{Security.} Among others qualities the security is an important concern in a live environment where the code is executed instantly. For this, we will design specific tests to assure the level of security of our system.

\item \textbf{Performance.} The performance of our system should scale for the generative design programs. The Rosetta tool will give us different \textit{backends} to test the performance of our tool. In fact, to be an interactive tool the response for an event should be less than one second. It imposes an restrict requirement because the time to generate an normal geometric model in a CAD \textit{backend} is by far more than one second. This issue, force us to establish a limit which the tool will be tested and compare this limit with other similar systems.

\item \textbf{Comparison with other systems.} We can only claim that our solution is somehow better than other if we compare both. Therefore we plan to compare our system with the existing programming environments for generative design. Thus the performance limit, stated above, will be our reference of comparison.
\end{itemize}