%!TEX root = ../report.tex

% 
% Evaluation
% 
\section{Evaluation}
\label{sec:eval}

The evaluation of the proposed architecture will be performed experimentally, building a prototype. The prototype will serve to test the proposed ideas as well as to evaluate them. To evaluate the prototype, we will use the Rosetta~\cite{lopes2011portable} generative design tool as a case study. As Rosetta is used by architects, and designer, we will receive real feedback from the target users. In this way we can evaluate if our prototype helps by correlating sketches with code or giving real time feedback to the users.

Furthermore, to evaluate our proposal we suggest the following evaluation metrics.

\begin{itemize}
\item \textbf{Correctness.} To assess the quality of our system we plan to test, individually, each proposed feature with an specific test case scenario, for example using the slider widget to explore the result of a parametric function and inserting different kinds of images and check if the image is well correlated with the function parameters. 

\item \textbf{Security.} Among others qualities, security is an important concern in a live environment where the code is executed instantly. In our case, the code is executed locally, however while the users are using the live code mode they can create dangerous constructs such as \texttt{eval}, \texttt{exec} and \texttt{file I/O} which can damage the operating system. On the other hand, in this mode is possible to block the environment with a simple ``while true''. To avoid these problems, we plan to implement sandboxing, similar to PythonTutor~\cite{GuoSIGCSE2013}, and design specific tests to test this feature.

\item \textbf{Performance.} The performance of our system should scale for the generative design programs. The Rosetta tool will give us different \textit{backends} to test the performance of our interactive tool. In fact, to be an interactive tool, the response for an event should be quick ($\sim$50ms). It imposes restrict requirements for the CADs tools, because these tools were designed for the speed of human operation, consequently they are the performance bottleneck. This issue, force us to establish a limit which this tool will be tested, thus we will compare this limit against other similar systems.

\item \textbf{Comparison with other systems.} We can only claim that our solution is somehow better than the other, if we compare them. Therefore, we plan to compare our system with the existing programming environments in generative design, particularly the visual environments, such as Grasshopper. Between these systems, the performance limit, stated above, will be our reference of comparison.
\end{itemize}